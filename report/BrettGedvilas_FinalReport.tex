\documentclass[11pt]{article}
\usepackage[top=1.5in, bottom=1in, left=1.25in, right=1.25in]{geometry}
\usepackage{graphicx}
%\usepackage{tikz}
%
%\usetikzlibrary{shapes.geometric, arrows}

\begin{document}


\title{Database Project Report}
\author{Brett Gedvilas\\CSCI 2421 Fall 2016}
\maketitle

\section*{Project Summary}

The goal of this project was to create a simple database in c++. Broadly speaking, a database is simply a structure that 
contains a set of (usually) related information. In our case, the database was to serve as an address book where each entry 
in the database can be considered a contact. Then, each entry contains information specific to that contact such as: name, 
address, phone number, email etc. The implementation of the database uses a binary search tree as the main data structure 
using a unique contact ID\# as the key for correctly inserting and traversing the tree. Moving deeper into the structure 
of the tree, each contact is an instance of an abstract data type called Record that holds the various fields for the 
contact.

After defining the main structure of the database, functionality had to be added to allow the user to make meaningful 
queries to the contents of the database and generate output.

\newpage



% ************************  Function Summary ***************************
\section*{Summary of Functions}

These functions all pertain to the user interaction with the database and demonstrate the functionality of the database. 
The parameters and some smaller functions have been omitted for clarity.

\begin{itemize}

\item searchTree()
\item subSearch() 
\item visitNodes()
\item modifyEntry()
\item removeEntry()
\item writeDatabase()
\item readFile()
\item sortLastName()
\item sortCompany()
\item sortState()
\item sortCountry()
\item sortCity()
\item selectFields()



\end{itemize}

\newpage


% ************************ Design Document Page ************************
\section*{Design Document}

\subsection*{C++ Implementation of an Address Book Database}
\begin{center}
\vspace{.5cm}
By: Brett Gedvilas\\
Student ID: 810-74-9234\\
Date: November 30, 2016\\
\end{center}

\subsection*{Problem Description}

	Being able to access and gain meaningful insights from large sets of data can be a very effective tool in a number of
different academic or professional fields. Unfortunately, a data set that includes thousands or millions of entries, each
having their own fields, presents a challenge of how to efficiently store, search, sort, and display the data as needed. 

	Our specific problem domain was to create a database that contains the contacts of an address book. This database must
satisfy a number of functional goals which allow the user to easily navigate the database and direct output of their choice
to a desired location. The program can read in a list of contacts from an external file and then allow the user to perform a variety of operations on the data such as: 
\begin{itemize}
	\item Modify the database manually by adding and deleting records in the database or modifying an existing entry.
	\item Search the database using a field of the users choice and perform sub-searches on the search results.
	\item Browse search results
	\item Sort the results from a database query using a given field.
	\item Display selected fields of search results.
	\item Output the database or search results to a file.
\end{itemize}
\newpage
\subsection*{Software Architecture}
\vspace{1cm}
	% ********************** Program Flow Chart ************************
Basic Program Flow	
	
\includegraphics[scale = .6]{flowChart.png}


\subsection*{Input Requirements}

	The inputs for the database come in three main flavours. 

	\begin{enumerate}
		\item The bulk of the data for each record will be read in from a file and directly parsed and stored in the main
		      binary search tree. The program will assume a specific file format for the records and any deviation from
		      this will cause the file to be rejected.
		
		\item The second input method is for the user to create a new record and directly add data to each field. In this
		   	  case, the program will perform basic input validation to ensure the user has entered appropriate values. 

		\item The third type of input will come from the user selecting menu items and performing searches on the database.
		  	  In the case of menu selection, the program will ask the user to select from a numeric list of menu options. 
		  	  The user must enter a valid integer corresponding to a menu option.	
		
	\end{enumerate}		
	

\subsection*{Output Requirements}

	The program can pipe information from the database to an output .txt file as ascii 
coded text. The format of the output mirrors the format of the input file for the database. Each record is separated by a $
|$. The fields of each record are newline delimited, and within the affiliate field, each affiliate is separated by a semi-
colon, and the sub-fields of each affiliate (if used) are delimited by a comma. The user will have the option of which 
fields to include in the output file. The output can consist of the entire database, or the results of a search query. A 
detailed description of each output field is shown below.

	\begin{description}
	
		\item[ID\#] A unique 9 digit unsigned integer. Stored as a c++ unsigned int data type. Any 9 digit integer is valid, 
					provided it is a non-zero positive value and does not match any existing ID\#. The ID\# 000000000 will
					be considered invalid. This gives the database up to 1 billion possible unique ID\#'s
					 
		\item[First Name] String
		\item[Middle Name] String
		\item[Last Name] String
		\item[Company Name] String
		\item[Home Phone] String
		\item[Office Phone] String
		\item[Email] String
		\item[Mobile Phone] String
		\item[Street Address] String
		\item[City] String
		\item[State] String
		\item[Zip Code] String
		\item[Country] String
		\item[Affiliate] Linked List - Each field of an affiliate (first name, last name, email and phone number) is stored
										as a string.
		
	\end{description}




\subsection*{Problem Solution Discussion}
	% 1 paragraph, ~100 words

The solution to this problem can be described in a number of steps. First, read in data from a file to populate the database. 
While the data is being read in, parse into contacts, insert each individual contact into a binary tree. To remove a contact: 
search binary tree for corresponding ID\# and delete the node holding that contact. To search database: get which field to 
search for and search term, visit every node in the binary tree and compare selected field to search term. If match is found,
 insert the contact information into a linked list. 


\subsection*{Data Structures}

	The main data structure used to organize the data base will take the form of a Binary Search Tree. Each record or contact
in the address book will be represented by a node in the tree, organized by the unique ID number of each contact. We can
leverage the power of the Binary Search Tree in order to efficiently search by ID number but searching using other fields
will require each node in the tree to be visited.
	
	Two main areas surface when considering the data structures to be used in the program: How to store the affiliates for
 each contact, and how to store search results from a database query.
	
	\begin{enumerate}
		\item I decided that the affiliates for each contact will be stored using a linked list, taking advantage of the c++ 
			  STL list container. Broadly speaking, a queue or stack does not translate well to this application. A user may
			  choose to access an affiliate not at the top of the stack (or front of queue) and that alone excludes them 
			  from consideration. In this case, I decided that an array or vector were also not appropriate for a few 
			  reasons. First, a c++ array is constricted to a certain size and the number of affiliates for a contact 
			  is unknown until runtime. Furthermore, just creating an array with a size that would likely be large enough 
			  for the number of affiliates would lead to a large amount of wasted memory. Even a contact with no affiliates
			  would still have an array of size n allocated for it. Similarly, while a vector is implemented as a dynamic 
			  array and would account for resizing to store more elements, they are still allocated with extra space. Again,
			  since I'm making the assumption that the number of affiliates will remain rather relatively small this leads 
			  to a fair amount of wasted memory. 
			  
			  The use of a linked list helps to alleviate some of these problems. Although a linked list is less efficient at
			  accessing its elements than an array or vector, because the number of entries will remain small the difference 
			  between an access time on the order of O(1) or constant, and an access time on the order of O(n) or linear will
			  be negligible. The trade off is one of speed vs. memory, but traversing a linked list of, say, 15 affiliates 
			  is trivial compared with the direct access afforded by a vector. That leads me to another consideration which
			  is how and when the affiliates will be accessed. Most of the access to affiliates will be done by the program
			  searching for a name or possibly a substring. In this case, each element will need to be visited anyway and 
			  so the number of times direct access to the affiliates is needed is assumed to be very low.
			  
			  Finally, using a linked list allows the container to grow only as much as is needed which helps to make 
			  efficient use of memory.
			  
		\item Search results from the database fall under one of two categories:
		
		\begin{enumerate}
			\item ID\# Search - Because the binary search tree is ordered using the ID numbers and each contact has a unique
								ID, any ID\# search will occur in log(n) time and return either a single Record, or nothing, 
								if the ID doesn't exist in the database.
			
			\item Other Fields - Data structures for other searches will by dynamically created and destroyed while the 
								 program is running. In my design, the user will not be limited to the number of searches
								 they can perform on the database. Saving a copy of every search result would create an
								 unreasonably large use of memory while the program was running, and in extreme cases could
								 deplete the available system memory and cause a crash. Instead, each instance of a search
								 will only exist for as long as the user is still making queries on that search data.
								 
								 The structure of each search result will once again be implemented as a linked list. For 
								 many of the same reasons as described above, a linked list once again maximizes our use 
								 of memory while reasonably managing the complexity of operating on the list. Any output of 
								 search results (to either a file or console) requires each item to be visited, so a linked
								 list performs no worse than any other structure in this respect. Using the c++ STL list 
								 allows for easy checks for the number of elements in the list, and if sorted, accessing
								 elements in alphabetical or reverse order.
		\end{enumerate}
			  		
		
	\end{enumerate}

\subsection*{User Interface Scheme}


	The program uses a menu-driven, text-based user interface scheme. The top level menu is the main menu, which then 
branches to sub-menus which allow the user to choose between different operations on the database. All menus require the
user to input an integer to select which operation to perform. The majority of menus are governed by a switch() 
statement to execute the correct menu selection. Some menu options call other menus, some cause functions to execute, and
some do both. The flow between menus is shown below.

\vspace{.25cm}

\includegraphics[scale = .6]{menuFlow.png}


\newpage


% ********************* Summary of Program Status *********************
\section*{Program Status}

The program currently performs all required functions as I interpret them except sub-sorting. The user can perform a primary 
sort on a list of search results, but the secondary sorting has not been implemented.


\subsection*{CSEgrid Status}

The program successfully compiles and runs on the CSE grid using both databasesmall.txt and databaselarge.txt.

\end{document}